\documentclass[a4paper, 12pt]{article}

\usepackage[english]{babel}
\usepackage{amsfonts, amsmath, amssymb, amsthm}
\usepackage{multicol, multirow}
\usepackage{graphicx}
\usepackage{tikz-cd}

\usepackage{hyperref}
\hypersetup{
  breaklinks,
  unicode,
  linktoc=all,
  bookmarksnumbered=true,
  bookmarksopen=true,
  pdfkeywords={ElegantBook},
  colorlinks,
  linkcolor=winered,
  citecolor=winered,
  urlcolor=winered,
  plainpages=false,
  pdfstartview=FitH,
  pdfborder={0 0 0},
  linktocpage
}


\usepackage{geometry}
\geometry{
  a4paper,
  top=25.4mm, bottom=25.4mm,
  left=20mm, right=20mm,
  headheight=2.17cm,
  headsep=4mm,
  footskip=12mm
}

\usepackage{xeCJK}
\usepackage{mathrsfs}
\usepackage{indentfirst,comment}
\usepackage{fontspec}


\usepackage[most]{tcolorbox}


\newtcbtheorem[auto counter, number within = section]{Theorem}{Theorem}{
  enhanced,
  colback=blue!5!white,
  colframe=blue!75!black,
  fonttitle=\bfseries,
  boxed title style={
    size=large,
    colback=blue!75!black, 
    colframe=blue!75!black,
  } 
}{thm}

\newtcbtheorem[auto counter, number within = section]{Proposition}{Proposition}{
  enhanced,
  colback=blue!5!white,
  colframe=blue!75!black,
  fonttitle=\bfseries,
  boxed title style={
    size=large,
    colback=blue!75!black, 
    colframe=blue!75!black,
  } 
}{proposition}

\newtcbtheorem[no counter]{Proof}{Proof}{
  enhanced,
  colback=white,
  colframe=blue!25,
  fonttitle=\bfseries,
  coltitle=black,
  boxed title style={
    size=large,
    colback=blue!25,
    colframe=blue!25,
  } 
}{prf}
\newtcbtheorem[no counter]{Remark}{Remark}{
  enhanced,
  colback=white,
  colframe=yellow!50!orange,
  fonttitle=\bfseries,
  coltitle=black,
  boxed title style={
    size=large,
    colback=blue!25,
    colframe=blue!25,
  } 
}{rmk}

\newtcbtheorem[auto counter, number within = section]{Definition}{Definition}{
  enhanced,
  colback=green!5!white,
  colframe=green!75!black,
  fonttitle=\bfseries,
  boxed title style={
    size=large,
    colback=blue!75!black, 
    colframe=blue!75!black,
  } 
}{def}

\newtcbtheorem[auto counter, number within = section]{Example}{Example}{
  enhanced,
  colback=white,
  colframe=gray,
  fonttitle=\bfseries,
}{exa}

\newcommand{\R}{\mathbb{R}}
\newcommand{\C}{\mathbb{C}}
\newcommand{\F}{\mathbb{F}}
\newcommand{\z}{\left}
\newcommand{\y}{\right}
\newcommand*{\defeq}{\stackrel{\text{def}}{=}}
\newcommand*{\isomap}{\stackrel{\sim}{\mapsto}}
\DeclareMathOperator{\image}{Im}

\usepackage[
backend=biber,
style=alphabetic,
sorting=ynt
]{biblatex}
%\addbibresource{ref.bib}

\title{抽象代数·续}

\date{March 2023}

\begin{document}

\maketitle

\section{Overview}
这个笔记面向的读者是计算机科学专业的读者,希望在理论计算机科学,程序语言等方向拥有数学基础,并且假定已经修读了数学系的本科生《抽象代数》课程,对群、环、模、域等结构和一些相关定理有一定认识。
由于课时的限制,许多教材会采用 M. Artin 的 Algebra 作为教材,作为入门固然不错,但是有所欠缺。
自行学习难免陷入递归学习,效率低下的烦恼,这就是本篇笔记的想要解决的问题。

\section{Category Theory}

范畴,泛性质(product, co-product, fiber, fiber co-product),函子

\section{Groups}
正合链,直和,自由 Abel 群,有限生成 Abel 群,群环

\subsection{Exact Sequence}
\begin{Definition}{正合列}{}
令一个同态序列
\[
G^\prime \stackrel{f}{\rightarrow} G \stackrel{g}{\rightarrow} G^{\prime\prime}
\]
是一个正合列,当且仅当 $\image f = \ker g$

\end{Definition}

下面我们给出一些正合链的重要例子:

\begin{Example}{}{}
对正合列
\[
H \stackrel{j}{\rightarrow} G \stackrel{\phi}{\rightarrow} G/H
\]
其中 $H$ 是 $G$ 的正规子群,同态 $j$ 是 inclusion, 同态 $\phi$ 是 canonical map.
有 $\image j = \ker \phi = H$.


下面再看这个两端为 $0$ 的正合列,
\[
0 \rightarrow G^\prime \stackrel{j}{\rightarrow} G \stackrel{\phi}{\rightarrow} G^{\prime\prime} \rightarrow 0
\]

注意到 $j$ 必定是 injective,$\phi$ 必定是 surjective.
如果应用到上面的正规子群的例子,我们有:
\[
0 \rightarrow H \rightarrow G \rightarrow G/H \rightarrow 0
\]
\end{Example}

同构定理以及正合列表示,略.

\subsection{Normal Subgroups}

\begin{Example}{$SL(F)$}{}
行列式是一个乘法群同态 $GL(\F) \mapsto \F$,这个同态的 kernel 是行列式为 1 的那些方阵,即 $SL(\F)$.
\end{Example}

\subsection{Abelian Groups}


\section{Modules}

半单性,模范畴

\section{Galois Theory}

\section{Homology}

\section{Algebraic Number Theory}

\printbibliography
\end{document}

